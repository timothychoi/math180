\documentclass[12pt]{article}
\usepackage[utf8]{inputenc}
\usepackage[letterpaper, portrait, margin=1in]{geometry}
\usepackage{amssymb}
\usepackage{amsmath}
\usepackage{graphicx}
\usepackage{tcolorbox}

\newcommand{\example}[1]{\noindent\textbf{Example #1\quad}}
\newcommand{\exercise}[1]{\noindent\textbf{Exercise #1\quad}}
\newcommand{\problem}[1]{\noindent\textbf{Problem #1\quad}}

\pagenumbering{arabic} % arabic, roman, Roman, alph, Alph, gobble
%\setcounter{page}{0} % sets the initial page

\setlength{\parindent}{0pt}
\setlength{\parskip}{10pt}
\setlength{\baselineskip}{15pt}
\linespread{1.2}

\font\sf = cmss10

\usepackage{fancyhdr}
\pagestyle{fancy}
\fancyhf{}
\lhead{MATH 180 Notes}
\rhead{Section 5.5: The Substitution Rule}
\rfoot{Page \thepage}
%\thispagestyle{empty}


\begin{document}

To differentiate the functions
$$ f(x) = \frac{1}{\sqrt{x^2 - 4x}}, \quad g(x) = e^{\sqrt{x}}, \quad \text{and} \quad h(x) = \sec^3(x) $$
we need to use the chain rule.
$$f'(x) = -\frac{x-2}{\sqrt{(x^2 - 4x)^3}}, \quad 
g'(x) = \frac{e^{\sqrt{x}}}{2\sqrt{x}}, \quad \text{and} \quad h'(x) = 3\sec^3(x)\tan(x) $$
If we need to find an antiderivative of a function that was obtained as a result of the chain rule could be challenging. For the function $g(x)$, there was practically no simplifying after using the chain rule. Hence, we could have guessed that $2\sqrt{x}$ came from the exponent $\sqrt{x}$. However, for $f$ and $h$, the expression after applying the chain rule has been simplified. So it might be hard to 

\vfill


Assigned Exercises: (p 399) 7 - 43 (odds), 45, 47, 55 - 65 (odds), 69, 73


\end{document}