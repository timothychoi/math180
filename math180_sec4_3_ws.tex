\documentclass[12pt]{article}
\usepackage[utf8]{inputenc}
\usepackage[letterpaper, portrait, margin=1in]{geometry}
\usepackage{amssymb}
\usepackage{amsmath}
\usepackage{graphicx}
\usepackage{tcolorbox}

\newcommand{\example}[1]{\noindent\textbf{Example #1\quad}}
\newcommand{\exercise}[1]{\noindent\textbf{Exercise #1\quad}}
\newcommand{\problem}[1]{\noindent\textbf{Problem #1\quad}}

\pagenumbering{arabic} % arabic, roman, Roman, alph, Alph, gobble
%\setcounter{page}{0} % sets the initial page

\setlength{\parindent}{0pt}
\setlength{\parskip}{10pt}
\setlength{\baselineskip}{15pt}
\linespread{1.2}

\font\sf = cmss10

\usepackage{fancyhdr}
\pagestyle{fancy}
\fancyhf{}
\lhead{MATH 180 Worksheet}
\rhead{Section 4.3: How Derivatives Affect the Shape of a Graph}
\rfoot{Page \thepage}
%\thispagestyle{empty}



\begin{document}

\exercise{1} Consider a function $f(x) = \frac{x}{x^2+1}$. 

\begin{tabular}{rl}
(a) & Find the intervals on which $f$ is increasing or decreasing. \\
(b) & Find the local maximum and minimum values of $f$. \\
(c) & Find the intervals of concavity and the inflection points.
\end{tabular}


\vspace{200pt}

\exercise{2} Consider a function $g(x) = x^4e^{-x}$. 

\begin{tabular}{rl}
(a) & Find the intervals on which $g$ is increasing or decreasing. \\
(b) & Find the local maximum and minimum values of $g$. \\
(c) & Find the intervals of concavity and the inflection points.
\end{tabular}

\pagebreak

\exercise{3} Sketch the graph of a function that satisfies the following conditions: (i) vertical asymptote $x = 0$; (ii) $f'(x) > 0$ if $x < -2$; (iii) $f'(x) < 0$ if $x > -2$ ($x \neq 0$); (iv) $f''(x) < 0$ if $x < 0$; (v) $f''(x) > 0$ if $x > 0$.

\vspace{200pt}

\exercise{4} Consider a function $h(x) = 5x^{2/3} - 2x^{5/3}$.

\begin{tabular}{rl}
(a) & Find the intervals on which $h$ is increasing or decreasing. \\
(b) & Find the local maximum and minimum values of $h$. \\
(c) & Find the intervals of concavity and the inflection points. \\
(d) & Use the information from parts (a)-(c) to sketch the graph.
\end{tabular}



\pagebreak

\exercise{5} Consider a function $p(x) = 









\end{document}