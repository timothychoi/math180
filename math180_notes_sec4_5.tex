\documentclass[12pt]{article}
\usepackage[utf8]{inputenc}
\usepackage[letterpaper, portrait, margin=1in]{geometry}
\usepackage{amssymb}
\usepackage{amsmath}
\usepackage{graphicx}
\usepackage{tcolorbox}

\newcommand{\example}[1]{\noindent\textbf{Example #1\quad}}
\newcommand{\exercise}[1]{\noindent\textbf{Exercise #1\quad}}
\newcommand{\problem}[1]{\noindent\textbf{Problem #1\quad}}

\pagenumbering{arabic} % arabic, roman, Roman, alph, Alph, gobble
%\setcounter{page}{0} % sets the initial page

\setlength{\parindent}{0pt}
\setlength{\parskip}{10pt}
\setlength{\baselineskip}{15pt}
\linespread{1.2}

\font\sf = cmss10

\usepackage{fancyhdr}
\pagestyle{fancy}
\fancyhf{}
\lhead{MATH 180 Notes}
\rhead{Section 4.5: Summary of Curve Sketching}
\rfoot{Page \thepage}
%\thispagestyle{empty}



\begin{document}

\subsubsection*{Guidelines for Sketching a Curve}

\begin{itemize}
\item [A.] \textbf{Domain}: Of course!
\item [B.] \textbf{Intercepts}: (i) $y$-intercept: $(0,f(0))$; (ii) $x$-intercept: Solve the equation $f(x) = 0$. Only the real solutions. Be aware that in most of cases you will not be able to solve this equation.
\item [C.] \textbf{Symmetry}: If $f$ is even, then we get the portion of the graph on the left hand side of $y$-axis
for free. The same is true for an odd function. We just need to flip the portion of the graph over the $x$-axis 
and then $y$-axis to get the portion on the left hand side. But this feature rarely occurs.
\item [D.] \textbf{Asymptotes}: (i) Horizontal Asymptote (HA) is found by finding the limits $\lim_{x \to \infty} f(x)$ and $\lim_{x \to -\infty} f(x)$; (ii) Vertical Asymptote (VA) happens at $x = a$ when $\lim_{x \to a^{\pm}} f(x) = \pm \infty$; (iii) Slant Asymptote (SA): The function of the form $f(x)/g(x)$ has a slant asymptote $y = mx + b$ if the quotient and the remainder obtained upon dividing $f(x)$ by $g(x)$ using the long division method is $mx$ and $b$ respectively.
\item [E.] \textbf{Intervals of Increase or Decrease}: Use the Increasing/Decreasing test.
\item [F.] \textbf{Local Maximum and Minimum Values}: Use the first derivative test.
\item [G.] \textbf{Concavity and Points of Inflection}: Use the Concavity test.
\item [H.] \textbf{Sketch the Curve}.
\end{itemize}




\end{document}